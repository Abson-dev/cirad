\documentclass[a4paper, oneside, 12pt]{book}

\usepackage[french]{babel} %permet de charger les attributs de la langue francaise
\usepackage{upgreek}
\usepackage{amsmath}
\usepackage{gensymb}
\usepackage[T1]{fontenc}
\usepackage[utf8]{inputenc} % activer les accents
\usepackage[left=2.5cm,right=2.5cm,top=2cm,bottom=3cm]{geometry} %pour changer les marges
\usepackage{color} %pour les couleurs avec \textcolor{color}{text}
\usepackage{xcolor} %fcolorbox 
\usepackage[tight]{shorttoc} %pour creer le sommaire
\usepackage{caption}
\newcommand{\sommaire}{\shorttoc{Sommaire}{0}}
\usepackage{titletoc}
\usepackage{multirow}
\usepackage{slashbox}
\usepackage{booktabs}
\usepackage{float}
\usepackage{pifont}
\usepackage{titlesec} %pour modifier la mise en forme des titres d'un document par \titleformat{command}[shape]{format}{label}{sep}{before-code}[after-code]
\setcounter{secnumdepth}{3} %pour choisir jusqu'à quel niveau on affiche la numérotation
\usepackage{amsmath} %pour les expressions mathématiques \underset \overset \xrightarrow \rightarrow
\usepackage{graphicx} %pour gerer tout ce qui est graphique
\usepackage{tikz, pgf}
\usetikzlibrary{calc}% pour faire des calculs
\usepackage{fancybox}
\definecolor{bleu}{rgb}{0.82, 0.1, 0.26}
%\definecolor{bleu}{rgb}{0.23,0.4,0.7}
\usepackage[most]{tcolorbox}
\usepackage{varwidth}
\usepackage{setspace} % pouvoir changer les interlignes (onehalfspace et doublespace)
\usepackage{slashbox} %permet de scinder un rectangle en deux \backslashbox{Texte dessous}{Texte dessus} 
\usepackage{array} %choisir son separateur pour les tableaux via !{} et @{}
\usepackage{colortbl} %pour colorer les tableaux grace a \columncolor{color} \rowcolor{color} \cellcolor{color}
\usepackage{tabularx}
\usepackage{booktabs}%pour des tableaux compliqués
\usepackage{soul} %pour souligner un texte
\usepackage{ulem} %souligner avec 2 traits
\usepackage{multirow}
\usepackage[tight]{shorttoc} %pour creer le sommaire
%\usepackage{pgf, tikz} % pour mes dessins latex


\newtcolorbox{boite}[2][]{enhanced,
	before skip=2mm, after skip=2mm,
	colback=white,colframe=black!50!white,boxrule=1pt,
	attach boxed title to top center={xshift=1mm,yshift*=1mm-\tcboxedtitleheight},
	varwidth boxed title*=-3cm,
	boxed title style={frame code={
		\path[fill=tcbcol@back!30!black]
		([yshift=-1mm,xshift=-1mm]frame.north west)
		 arc[start angle=0,end angle=180, radius=1mm]
	 	([yshift=-1mm,xshift=1mm]frame.north east)
 		 arc[start angle=180, end angle=0,radius=1mm];
 	 	\path[left color=tcbcol@back,right color=tcbcol@back!60!black,
 	 	 middle color=tcbcol@back!80!black]
  	 	([xshift=-2mm]frame.north west) -- ([xshift=2mm]frame.north east)
   		[rounded corners=1mm]-- ([xshift=1mm,yshift=-1mm]frame.north east)
   		-- (frame.south east) -- (frame.south west)
   		-- ([xshift=-1mm,yshift=-1mm]frame.north west)
   		[sharp corners]-- cycle;
   		},interior engine=empty,
   		},
   	fonttitle=\bfseries,
   	title={#2},#1}

\newtcolorbox{tableau}[2][]{enhanced, fonttitle=\bfseries\large,fontupper=\normalsize,
	colback=bleu!10!white,colframe=bleu!70!white,colbacktitle=bleu!70!white,
	coltitle=black,center title}

\newtcolorbox{obj}{enhanced jigsaw, colframe=red!50!white, colback=bleu!10!white, breakable, before skip=10pt, after skip=10pt}
\newtcolorbox{definition}{ enhanced jigsaw, colframe=bleu, interior hidden, breakable, before skip=10pt, after skip=10pt}

\newtcolorbox{encadre}[2][]{enhanced,skin=enhancedlast jigsaw,
	attach boxed title to top left={xshift=-4mm,yshift=-0.5mm},
	fonttitle=\bfseries\sffamily,varwidth boxed title=0.7\linewidth,
	colbacktitle=bleu!75!black,colframe=red!70!black,
	interior style={top color=bleu!2!white,bottom color=red!2!white},
	boxed title style={empty,arc=0pt,outer arc=0pt,boxrule=0pt},subtitle style={boxrule=0.4pt,
		colback=bleu!50!red!25!white},
	underlay boxed title={
		\fill[bleu!85!white] (title.north west) -- (title.north east)
		-- +(\tcboxedtitleheight-1mm,-\tcboxedtitleheight+1mm)
		-- ([xshift=4mm,yshift=0.5mm]frame.north east) -- +(0mm,-1mm)
		-- (title.south west) -- cycle;
		\fill[blue!45!white!50!black] ([yshift=-0.5mm]frame.north west)
		-- +(-0.4,0) -- +(0,-0.3) -- cycle;
		\fill[blue!45!white!50!black] ([yshift=-0.5mm]frame.north east)
		-- +(0,-0.3) -- +(0.4,0) -- cycle; },
	title={#2},#1}

\usepackage{fourier-orns} % pour la ligne haut de page
\usepackage{pstricks} %
\usepackage{fancybox}
\usepackage{fancyhdr}
\pagestyle{fancy}

%%%%%%%%%%%%%%%%%%%graphique et tableau
\begin{comment}
\usepackage{array}
\usepackage{multirow}
\usepackage{graphicx}
\usepackage{colortbl}
\definecolor{bluecell}{rgb}{0.65,0.77,0.82}
\definecolor{graycell}{rgb}{0.933333333333,0.933333333333,0.933333333333}
\definecolor{graytitle}{rgb}{0.81568627451,0.811764705882,0.83137254902}
\end{comment}
%%%%%%%%%%%%%%%%%%%


%%%%%%%%%   On paramètre le style des chapitres, sections et autres  %%%%%%%%%
\newlength\chaptaille   % Pour la taille du chapitre
\settowidth\chaptaille{\huge\chaptertitlename}
%\titleformat{command}[shape]{format}{label}{sep}{before-code}[after-code]
\titleformat{\chapter}[display]
{\normalfont\filcenter\bfseries\large}
{\tikz[remember picture,overlay]
	{
		\node[fill=bleu,font=\fontsize{60}{72}\selectfont\color{white},anchor=north east,minimum size=\chaptaille] 
		at ([xshift=-15pt,yshift=-94pt]current page.north east) 
		(numb) {\thechapter}; % Pour le num du chapitre à droite dans le carré
		\node[rotate=0,anchor=south,inner sep=0pt,font=\Large] at (numb.north) {\chaptertitlename};
	}
}{0pt}{\fontsize{20}{25}\selectfont\color{bleu}\rule{\linewidth}{5pt}\\}[\vskip5pt\large\hrule\vskip20pt]  % Vskippr la distance entre le titre du chapitre et le shorttable of contents
\titlespacing*{\chapter} % espace entre la bordure haute, droite de la page et le début du titre du chapitre
{0pt}{0pt}{2pt}

\titleformat{\section}
{\color{bleu!75!black}\usefont{T1}{cmr}{l}{sc}\bfseries}
{}
{0em}
{\thesection\quad \Large}
[]

\titleformat{\subsection}
{\color{bleu!75!black}\large\usefont{T1}{ptm}{l}{n}}
{}
{0em}
{\thesubsection\quad\large}
[]

\titleformat*{\subsubsection}{\bfseries\color{bleu!75!black}}
 

\definecolor{myblue}{rgb}{0.82, 0.1, 0.26}

\renewcommand{\thesection}{\Roman{section}}
\renewcommand{\thesubsection}{\Roman{section}.\arabic{subsection}}

%%%%%%%%%   On paramètre l'entête et le pied de page   %%%%%%%%%%%%
\begin{comment}
\fancyfoot[C]{} 
\fancyfoot[L]{\fontfamily{pzc}\itshape \small TOSSOU Comlan Rodrigue \\ Elève Ingénieur des Travaux Statistiques en 3\ieme{} année}
\fancyfoot[R]{{\small \itshape \bfseries page\\{\doublebox{\bfseries \thepage}}}}
%\renewcommand{\footrulewidth}{2pt}
%\fancyhead[C]{{\small \centering \leftmark}} pour la ligne avec design ci dessous
\fancyhead[C]{} 
\fancyhead[L]{}
\fancyhead[R]{}
\end{comment}

%pour les figures
\newtcolorbox[blend into=figures]{myfigure}[2][]{float=htb,capture=hbox,
	blend before title code={\fbox{##1}\ },title={#2},every float=\centering,#1}
%%%%%%%%%%%%%%%%%%% AVEC FANCY

%\begin{comment}
\usepackage{fancyhdr}
\pagestyle{fancy}
\lhead{}
\chead{\small{Modélisation de la distribution des principales espèces ligneuses dans deux parcs agroforestiers du bassin arachider sénégalais.}}
\rhead{}
\lfoot{\fontfamily{sans-serif} \small Aboubacar HEMA \\ Elève Ingénieur des Travaux Statistiques en 4\ieme{} année}
\fancyfoot[R]{{\small  page\\{\shabox{\bfseries \thepage}}}}
\cfoot{}
\rfoot{\thepage}
\renewcommand{\headrulewidth}{0.8pt}
\renewcommand{\footrulewidth}{0.8pt}
%\end{comment}



%%% ligne horizontale avec design
\begin{comment}
\newcommand{\myRule}[3][black]{\textcolor{#1}{\rule{#2}{#3}}} % ligne horizontale avec couleur, longueur et largeur
\setlength{\fboxrule }{1pt} %pr l'épaisseur des bordures de boite
\renewcommand\headrule{\myRule[black]{7cm}{2pt}
	\raisebox{-2.1pt}[10pt][10pt]{\quad\aldineright\decosix\aldineleft\quad\par\bigskip}\myRule[black]{7cm}{2pt}}
\renewcommand{\headrulewidth}{1pt}
\end{comment}

\renewcommand{\tablename}{Tableau ~} % pour changer le titre des tableaux

\newcolumntype{Y}{>{\raggedleft\arraybackslash}X}%pour modifier le format d'une colone		

% Pour les petites tables de matières de chaque chapitre
\newcommand{\ChapCont}{%
	\startcontents[chapters]
	\printcontents[chapters]{}{1}{\noindent{\color{bleu}\rule{\textwidth}{2pt}}\par\medskip}
}

\onehalfspacing
%\hyphenpenalty=10000 % empêche les coupures de mots
\usepackage{hyphenat}% pour éviter la coupure de certains bouts de texte avec \nohyphens



%%%%%%%%%%%%%%%%%%%%%%%%%%%%%%%%%%%%%%%%%%%
\usetikzlibrary{shapes}
\usetikzlibrary{trees}
%\usepackage[svgnames]{xcolor}
	\tcbuselibrary{skins,theorems,breakable}
\xdefinecolor{mycol}{named}{myblue}
\tcbset{
		mybox/.style={
			breakable,
			enhanced standard,
			boxrule=0.4pt,titlerule=-0.2pt,drop fuzzy shadow,
			width=\linewidth,
			title style={top color=mycol!30,bottom color=mycol!0.5},
			overlay unbroken and first={
				\path[fill=mycol]
				([xshift=5pt,yshift=-\pgflinewidth]frame.north west) to[out=0,in=180] ([xshift=20pt,yshift=-5pt]title.south west) -- ([xshift=-20pt,yshift=-5pt]title.south east) to[out=0,in=180] ([xshift=-5pt,yshift=-\pgflinewidth]frame.north east) -- cycle;
			},
			fonttitle=\Large\bfseries\sffamily,
			fontupper=\sffamily,
			fontlower=\sffamily,
			before=\par\medskip\noindent,
			after=\par\medskip,
			center title,
			toptitle=3pt,
			top=11pt,topsep at break=-5pt,
			colback=white
		}}
\newtcolorbox{MyBlock}[2][\linewidth]{mybox,width=#1,title=#2}
%%%%%%%%%%%%%%%%%%%%%%%%%%%%

\newcommand{\HRule}{\rule{\linewidth}{0.5mm}}%% deuxieme page
%%%%%%%%%%%%%%%%%%pour les théoremes utilisés

\newtcbtheorem[auto counter,number within=section]{theo}%
  {Theorem}{fonttitle=\bfseries\upshape, fontupper=\slshape,
     arc=0mm, colback=blue!5!white,colframe=blue!75!black}{theorem}

%%%%%%%%%%%%%%%%%%%%%


\def\permille{\ensuremath{{}^\text{o}\mkern-5mu/\mkern-3mu_\text{oo}}} %% definir pourmille
\usepackage{times}
\begin{document}



\thispagestyle{empty}
\begin{comment}
\begin{center}
	
	\begin{minipage}[t]{0.4\textwidth}
	\begin{center}
	\textcolor{black}{ \textbf{REPUBLIQUE DU SENEGAL}}\\
	\vspace*{0.2cm}
	\includegraphics[height=2.3cm, width=2.3cm]{senegal}
	%$$ \textbf{************}$$
	$$ \textbf{\textit{ \tiny Un Peuple - Un But - Une Foi}}$$
	
	\textcolor{black}{\textbf{MINISTERE DE L'ECONOMIE, DU PLAN ET DE LA COOPÉRATION}}\\
		\end{center}
		\end{minipage}
		\hspace{3cm}
		\begin{minipage}[t]{0.4\textwidth}
		\begin{center}
		\textcolor{black}{ \textbf{BURKINA FASO}}\\
	\vspace*{0.2cm}
	\includegraphics[height=2.3cm, width=2.3cm]{bf}
	%$$ \textbf{************}$$
	$$ \textbf{\textit{ \tiny Unité - Progrès - Justice}}$$
	
	\textcolor{black}{\textbf{MINISTERE DE L'ECONOMIE ET DES FINANCES}}\\
		\end{center}
		\end{minipage}	
	
	
	\begin{minipage}[t]{0.4\textwidth}
	\begin{center}
	%\hspace{-3cm}
		%\begin{figure}[H]
		\centering{
		\includegraphics[height=2.1cm, width=2.8cm]{ansd}}
		%\end{figure}
		
		\textcolor{black}{\textbf{AGENCE NATIONALE DE LA STATISTIQUE ET DE LA DÉMOGRAPHIE}}\\
		\end{center}
		\end{minipage}
		\hspace{3cm}
		\begin{minipage}[t]{0.4\textwidth}
		\begin{center}
		%\begin{figure}[H]
		\centering{
		\includegraphics[height=2.1cm, width=2.8cm]{insd}}
		%\end{figure}
		
		\textcolor{black}{\textbf{INSTITUT NATIONALE DE LA STATISTIQUE ET DE LA DÉMOGRAPHIE}}\\
	
		\end{center}
		\end{minipage}
		$$ \textbf{************}$$
		\begin{center}
		%\begin{figure}[H]
		\centering{
		\includegraphics[height=2cm, width=2.3cm]{ensae}}
		%\end{figure}

		\vspace*{0.051cm}
		\textcolor{black}{\textbf{ÉCOLE NATIONALE DE LA STATISTIQUE ET L'ANALYSE ÉCONOMIQUE}}\\
	
		\end{center}
		
	\hspace*{-1.5cm}
	\begin{tikzpicture}[overlay, remember picture]
	\hspace{1,5cm}
\node[fancytitle2, right=12pt, rounded corners] at (-2,0) {\bfseries \footnotesize RAPPORT DE STAGE ITS 2};
\end{tikzpicture}%
	\end{center}
	\begin{center}
	\begin{tikzpicture}[overlay, remember picture]
	\node (A) at (-8,0) {};
	\node (B) at (9,0) {};
	\draw [myblue, scale=2] (A) to [ornament=88] (B);
	\end{tikzpicture}
\end{center}
\begin{center}
\hspace*{0.5cm}
		\textcolor{black}{\textbf{ COMPRENDRE LA MORTALITÉ DES ENFANTS DE moins d'un an AU SAHEL : RÔLE DES MÈRES }}
	\end{center}
	\begin{center}
	\begin{tikzpicture}[overlay, remember picture]
	\node (A) at (-8,0) {};
	\node (B) at (9,0) {};
	\draw [myblue, scale=2] (A) to [ornament=88] (B);
	\end{tikzpicture}
\end{center}	
\begin{center}
\hspace*{0cm}
	\textcolor{myblue}{\textbf{Réalisé par}}\\
	\vspace*{0,2cm}
	\hspace*{0cm}
	\begin{tabular}{cc}
	\hline\hline\\	
	\emph{\textbf{BAMOUNI Christophe Jean-Baptiste}}\\
	\\
	
	Elève Ingénieur des Travaux Statistiques, ITS 3\\
	\hline\hline
	\end{tabular}\\
	\vspace*{0,4cm}	
	\hspace*{0cm}
    \textcolor{myblue}{\textbf{\textsc{maitre de stage :}}}\\
    \vspace*{0,1cm}
    \hspace*{0cm}
    \begin{tabular}{cc}
	\hline\hline\\
	
\emph{\textbf{M. Nouffou SAWADOGO}}\\
\\

   \textsc{Démographe à l'INSD}
\\
 \hline\hline  
 \end{tabular}\\
 \vspace*{0,1cm}

\end{center}
\begin{center}
\textcolor{myblue}{\textbf{Août 2019}}
\end{center}
\begin{center}
		\begin{tikzpicture}[overlay, remember picture]
		\draw[line width=2.5pt, color=myblue] (-9.5, 26) rectangle (9.5,-1); %% [line width=5pt], [very thick]
		\draw[line width=2.5pt, color=myblue] (-9.2, 25.7) rectangle (9.2,-0.7);
		\end{tikzpicture}
\end{center}
\end{comment}
	\newpage
%\onehalfspacing
%%%%%%%%%%%%%
\pagenumbering{roman} %numerotation

%\renewcommand\familydefault{lmss}

\onehalfspacing
%****************deuxieme page de garde*****************
\begin{center} \large
\textcolor{bleu}{{\Huge \textit{Modélisation de la distribution des principales espèces ligneuses dans deux parcs agroforestiers
			du bassin arachidier sénégalais.}}\\
\HRule}%% Hrule est une macro
\vfill
\textit{Document rédigé par :}\\
\textbf{Aboubacar HEMA}\\
\textit{Élève Ingénieur des Travaux Statistiques, ITS4.}
\vfill
\textit{Sous l'encadrement de :}\\
\textbf{ok}\\
\textit{ok} \\

\vfill
\end{center}


\newpage
\vspace*{10cm}
\begin{MyBlock}{AVERTISSEMENT}
ok

\end{MyBlock}


\newpage
\vspace*{1cm}

\begin{tcolorbox}[enhanced,attach boxed title to top center={yshift=-3mm,yshifttext=-1mm},
  colback=blue!5!white,colframe=blue!75!black,colbacktitle=red!80!black,
  title=AVANT-PROPOS,fonttitle=\bfseries,
  boxed title style={size=small,colframe=red!50!black} ]
ok
\end{tcolorbox}



\newpage
\begin{flushright}
\Large{\og  \textit{\textcolor{bleu}{Ressentir de la gratitude et ne pas l'exprimer, c'est comme emballer un cadeau et ne pas le donner.}} \fg{} \\ \textbf{\textcolor{blue}{William Arthur Ward}} \\ }
\end{flushright}





%% pour le sommaire	
{\setlength{\baselineskip}{1.0\baselineskip}	
	\shorttoc{Sommaire}{1}
	\par} 
	\normalsize
	%\addcontentsline{toc}{chapter}{SOMMAIRE}
	%\tableofcontents


\newpage
\listoffigures
\newpage
\listoftables
\newpage
\chapter*{Liste des sigles}
\vspace*{4cm}

\begin{tcolorbox}[colback=red!5!white,colframe=red!75!black]
ok
\end{tcolorbox}



\newpage
\chapter*{RÉSUMÉ}
\vspace*{4cm}
ok

\newpage
\chapter*{INTRODUCTION GÉNÉRALE }
\setcounter{page}{1}
\pagenumbering{arabic}

ok

\part{CADRE D’ÉTUDE ET MÉTHODOLOGIE}

\chapter{Présentation de la zone d'étude, de la source des données }
ok

\section{Zone d’étude}
L'étude est menée dans le bassin arachidier du Sénégal, sur une superficie de 20x20km centrée sur la commune de Ngayokhème (Département de Fatick). La zone est caractérisée par un système agricole à base d'arbres dominé par Faidherbia albida. En effet, le Bassin arachidier du Sénégal couvre le centre ouest du pays et est dominé par des sols ferrugineux tropicaux, permettant une production agricole composée essentiellement de céréales sèches (mil) et de légumineuses (arachide, niébé). Toutefois, avec près de 60\% de la population rurale, la zone fait face à une forte pression démographique, la réduction du temps de jachère et l’insuffisance voire l’absence de la fertilisation des terres, conduisant à la destruction du couvert végétal et l’érosion de la biodiversité.
\begin{figure}[H]
	\centering
	\caption{Présentation de la zone d'étude}
	\includegraphics[height=10cm]{etude}	
\end{figure}

\subsection{Cartographie du gradient d’hétérogénéité du paysage}
Dans un premier temps, nous avons procédé à la représentation spatialisée du gradient d’hétérogénéité du paysage. Il s’agissait de stratifier la zone d’étude en classes de paysage agroécologiques homogènes (en interne) mais différentes entre elles de par leurs caractéristiques agricoles et écologiques. 
\begin{figure}[H]
	\centering
	\caption{Etapes de cartographie du gradient d’hétérogénéité du paysage}
	\includegraphics[height=4cm]{ndao}	
\end{figure}
\subsubsection{Segmentation du paysage}
Une série temporelle de 13 images Sentinel-2, de janvier à octobre 2017, couvrant la saison sèche et la saison humide a été utilisée. Une segmentation multirésolution est effectuée sur la série temporelle de NDVI dérivée des images Sentinel pour identifier et délimiter des unités homogènes en termes de fonctionnement de la végétation (y compris les cultures). La zone d’étude a été ainsi segmentée en 668 unités de paysage agricole avec des superficies allant de 10,61 ha (plus petite unité) à 489,68 ha (plus grande unité). 
\begin{figure}[H]
	\centering
	\caption{Segmentation de la zone d’étude en unités de paysage agricole (Représentation sur fond d’image Pléiades THRS)}
	\includegraphics[height=12cm]{upi}	
\end{figure}
\subsubsection{Choix des variables de stratification}
En supposant que les paysages agricoles sont structurés par leurs conditions environnementales notamment les facteurs écoclimatiques, nous avons identifié 5 variables caractéristiques et structurantes pour la biodiversité et la productivité végétale. Il s’agit de :
\begin{itemize}
	\item deux variables écophysiologiques, la productivité de la végétation et sa dynamique durant la période 2000-2015 ;
	\item une variable agroclimatologique, l’évapotranspiration réelle ; 
	\item une variable pédologique, le type de sol ;
	\item une variable écologique, la couverture ligneuse.
\end{itemize}
Les valeurs de ces variables ont été dérivées de différentes sources de données géospatiales.	
\subsubsection{Calcul des variables de stratification}
Pour chacune des unités paysagères issues de la segmentation la moyenne et l’écart type des variables biophysiques retenues, caractéristiques de la diversité et de la structure du paysage agricole sont calculées. La valeur moyenne a été choisie comme mesure de la tendance centrale tandis que les valeurs d'écart-type ont été considérées comme un indicateur d'homogénéité spatiale à l'intérieur de chaque unité de paysage, ce qui signifie une homogénéité en termes de configuration et de composition du paysage.
\subsubsection{Classification}
Les unités de paysage ont été classées en fonction des 5 variables de stratification. Une méthode de classification ascendante hiérarchique sur composantes principales (HCPC : Hierarchical Clustering on Principal Components) a été utilisée à cet effet.
Le résultat est un gradient d’hétérogénéité du paysage divisé en 4 classes représentatives des grandes formes et configurations du paysage.
\begin{figure}[H]
	\centering
	\caption{Carte du gradient d’hétérogénéité du paysage (Zone d’étude)}
	\includegraphics[height=12cm]{strat_zone}	
\end{figure}
\section{Sources des données}
\subsection{Données environnementales}
Deux campagnes de terrain pour collecter des données géoréférencées sur le paysage ont été réalisées. La première pour l’inventaire des espèces arborées s’est déroulée du 22 juin au 02 juillet 2018 et la seconde pour la collecte de données sur l’occupation du sol et les cultures en particulier est effectuée du 05 au 10 septembre 2018.\\

Sur la base de la carte du gradient d’hétérogénéité du paysage déjà élaborée, un protocole d'échantillonnage spatial optimisé a été produit pour couvrir régulièrement toutes la diversité spatiale de la zone d’étude. C’est un plan d’échantillonnage stratifié pondéré en fonction du gradient d’hétérogénéité du paysage. Ainsi, 213 sites d’observation ont été définis pour la première campagne et 45 sites pour la seconde.
\begin{figure}[H]
	\centering
	\caption{Répartition des sites d’échantillonnage sur le terrain}
	\includegraphics[height=12cm]{arbres}	
\end{figure}
A l’issue de la première campagne 9258 arbres répartis en 63 espèces ont été identifiés et géoréférencés.\\

La seconde campagne a permis de digitaliser 734 polygones d’occupation/utilisation du sol.

\textbf{Cartographie du système agroforestier}\\
\begin{itemize}
	\item \textbf{Acquisition d’images satellites haute et très haute résolution spatiale}\\
	Trois images satellitaires Pléiades (très haute résolution spatiale) ont été commandées : en fin de saison sèche (mai), en saison des cultures (octobre) et en début de saison sèche (décembre). Les images de mai et de décembre ont été reçue mais celle d’octobre n’a pas été acquise du fait d’une couverture nuageuse trop défavorable.
	Une série temporelle d’images PlanetScope (très haute résolution spatiale) et une série temporelle d’images Sentinel 2 ont été téléchargées durant la période juin-novembre 2018 (couvrant la saison culturale).
	\item \textbf{Traitement des données et cartographie de l’occupation/utilisation du sol}\\
	La chaine de traitement d’images satellites MORINGA a été utilisée pour réaliser la cartographie de l’occupation/utilisation du sol. MORINGA est une chaine de traitement d’images satellitaires développée par des chercheurs du Cirad.  Elle fait appel à des fonctions de l’Orfeo Tool Box (OTB) et permet de produire une carte d’occupation du sol, à partir d’une image THRS, d’une série temporelle d’images, et d’une base de données d’apprentissage (polygones dont l’occupation du sol est connue).\\
	Le résultat est une carte d’occupation/utilisation du sol avec 9 classes.
	\begin{figure}[H]
		\centering
		\caption{Carte d'occupation/utilisation du sol de la zone d'étude}
		\includegraphics[height=12cm]{occupe_sol}	
	\end{figure}
\end{itemize}
\subsection{Données bioclimatiques : WorldClim - Données climatiques mondiales}
WorldClim est un ensemble de couches climatiques mondiales (données climatiques maillées) avec une résolution spatiale d'environ  1 $km^2$. Ces données peuvent être utilisées pour la cartographie et la modélisation spatiale.\\
Dans le cadre de notre étude, nous avons utilisé la version 2.0 pour télécharger les données bioclimatiques.\\

Les variables bioclimatiques dérivent des valeurs mensuelles de la température et des précipitations pour générer des variables plus significatives sur le plan biologique. Utilisées dans la modélisation de la répartition des espèces et les techniques de modélisation écologique, les variables bioclimatiques représentent les tendances annuelles (la température annuelle moyenne, les précipitations annuelles, …), la saisonnalité (par exemple, la plage annuelle de température et de précipitations) et les facteurs environnementaux extrêmes ou limitants (par exemple, la température du mois le plus froid et le plus chaud, et les précipitations des quartiers secs).


\chapter{Révue de la littérature}
\section{Modèles de distribution des espèces}
\section{Caractéristiques générales des espèces étudiées.}
%%%%%%%%%%%%%%%%%%%%%%%%%%%%%%%%%%%%%%%%% Faidherbia albida
\subsection{Faidherbia albida}
%%%%%%%%%%%%%%%%%%%
\subsection{Balanites aegyptiaca}
%%%%%%%%%%%%%%%%%%%%%%%%%%%%%%%%%%%%%%%%% Balanites aegyptiaca %%%%%%%%%%%%%%%%%%%
\subsection{Anogeissus leiocarpus}
%%%%%%%%%%%%%%%%%%%%%%%%%%%%%%%%%%%%%%%%% Anogeissus leiocarpus %%%%%%%%%%%%%%%%%%%
\subsection{Adansonia digitata}
%%%%%%%%%%%%%%%%%%%%%%%%%%%%%%%%%%%%%%%%% Adansonia digitata %%%%%%%%%%%%%%%%%%%
\subsection{Acacia nilotica}
%%%%%%%%%%%%%%%%%%%%%%%%%%%%%%%%%%%%%%%%% Acacia nilotica %%%%%%%%%%%%%%%%%%%
\section{Les variables à haute résolution spatiale}
\section{Changements climatiques au Sénégal}


\chapter{Approche méthodologique et outils d'analyse}

\part{Analyse spatiale  et modélisation de la distribution des espèces}
\chapter{Analyse spatiale descriptive }
\section{Analyse spatiale descriptive dans la zone d'étude}

\begin{figure}[H]
	\centering
	\caption{Les types de sol dans la zone d'étude}
	\includegraphics[height=10cm]{t_sol}	
\end{figure}
La zone d'étude est caractérisée par trois types de sols:
\begin{itemize}
	\item ferrugineux tropicaux, dont les caractéristiques sont les suivantes:
	\begin{itemize}
		\item profondeur de sol variable (en tout cas moins de 3 m);
		\item une nette horizonation des profils pédologiques;
		\item horizon de surface riche en matière organique:1,5 pour cent dans la partie supérieure, moins de 1 pour cent dans sa partie inférieure;
		\item présence, parfois, d'un horizon de transition lessivé (30-40 cm) et d'un horizon d'accumulation en argile et en fer avec ou sans concrétions et nodules;
		\item le rapport limon fin sur argile est toujours supérieur à 0,20 et celui SiO2 / Al2O3 à 2;
		\item le drainage interne généralement bon est parfois mauvais;
		\item présence d'horizon induré (surtout dans les bas-versants).
	\end{itemize}
\item hydromorphes, qui montre des marques physiques de saturation régulière en eau;
\item hydromorphe salé
\end{itemize}
\begin{figure}[H]
	\centering
	\caption{Les types de sol suivant les zones}
	\includegraphics[height=10cm]{sol_zones}	
\end{figure}
L’analyse montre que :
\begin{itemize}
	\item la zone 1 correspond aux unités paysagères caractérisées par une couverture homogène et une faible densité et diversité des arbres, avec des sols hydromorphes salés ;
	\item la zone 2 est principalement caractérisée par un important changement positif de la végétation au cours des 15 dernières années, avec des sols généralement ferrugineux tropicaux et rarement hydromorphes ;
	\item la zone 3 diffère de la seconde par une plus grande variabilité spatiale signifiant une plus grande hétérogénéité du faciès de la végétation et également par une plus grande fréquence de sols hydromophes ;
	\item la quatrième présente une productivité végétale et une densité de couverture ligneuse plus élevées, mais aussi une plus faible variabilité spatiale. Les sols y sont en général hydromorphes.	
\end{itemize}
\begin{figure}[H]
	\centering
	\caption{Pourcentage d'espèces ligneuses par zones}
	\includegraphics[height=9cm]{arbre_zone}	
\end{figure}
\section{Analyse spatiale descriptive des espèces étudiées}
\subsection{Faidherbia albida}
Les deux derniers graphiques ont la même source de données, montrant la distribution de la présence(1)/absence(0) de l'espèce \textbf{Faidherbia albida} en termes de zone. Celui de gauche montre en pourcentage, tandis que celui de droite montre en valeur absolue.\\
La zone semble être un \textbf{bon prédicteur}, car la probabilité de présence/absence de l'espèce \textbf{Faidherbia albida} est différente compte tenu des zones. Elle donne un ordre aux données .\\
De la prémière parcelle(\%), la \textbf{probabilité} de présence de \textbf{Faidherbia albida} pour la \textbf{zone 3} est de \textbf{45.1\%}, tandis que pour les zones 1, 2 et 4 ,respectivement elle est de 23.3\%, 42.2\% et 39.6\%. \\
La probabilité de présence de \textbf{Faidherbia albida} dans la \textbf{zone 3} est plus élévée que dans les autres zones(45.1\% contre 23.3\%, 42.2\%, 39.6\%, respectivement).
De la deuxième parcelle(nombre):
\begin{itemize}
	\item Il y a un total de 193 arbres dans la zone 1:
	\begin{itemize}
		\item 45 d'entre eux répresentent \textbf{Faidherbia albida}(45/193=23.3\%);
		\item les 148 répresentent les autres arbres.
	\end{itemize}
	\item Il y a un total de 2646 arbres dans la zone 2:
	\begin{itemize}
		\item 1117 d'entre eux répresentent \textbf{Faidherbia albida}(1117/2646=42.2\%);
		\item les 1529 répresentent les autres arbres.
	\end{itemize}
	\item Il y a un total de 3048 arbres dans la zone 3:
	\begin{itemize}
		\item 1375 d'entre eux répresentent \textbf{Faidherbia albida}(1375/3048=45.1\%);
		\item les 1673 répresentent les autres arbres.
	\end{itemize}
	\item Il y a un total de 3371 arbres dans la zone 4:
	\begin{itemize}
		\item 1335 d'entre eux répresentent \textbf{Faidherbia albida}(1335/3371=39.6\%);
		\item les 2036 répresentent les arbres.
	\end{itemize}
\end{itemize}





\begin{figure}[H]
	\centering
	\caption{Probabilité de présence de Faidherbia albida dans les différentes zones}
	\includegraphics[height=9cm]{Zone_Ferdhbia}	
\end{figure}
\begin{figure}[H]
	\centering
	\caption{Distribution de Faidherbia albida suivant la zone 1}
	\includegraphics[height=10cm]{PA_FaidherbiaZ1}	
\end{figure}
\begin{figure}[H]
	\centering
	\caption{Distribution de Faidherbia albida suivant la zone 2}
	\includegraphics[height=10cm]{PA_FaidherbiaZ2}	
\end{figure}
\begin{figure}[H]
	\centering
	\caption{Distribution de Faidherbia albida suivant la zone 3}
	\includegraphics[height=10cm]{PA_FaidherbiaZ3}	
\end{figure}
\begin{figure}[H]
	\centering
	\caption{Distribution de Faidherbia albida suivant la zone 4}
	\includegraphics[height=10cm]{PA_FaidherbiaZ4}	
\end{figure}
\begin{figure}[H]
	\centering
	\caption{Probabilité de présence de Faidherbia albida suivant productivité végétale(regroupé en 3 classes)}
	\includegraphics[height=10cm]{NDVI_mean_groupe_Faidherbia_3}
\end{figure}
 Ce nouveau graphique basé sur le regroupement en trois classes de l'indice de végétation montre clairement comment la probabilité de présence de \textbf{Ferdherbia albida} diminue  à mesure que l'indice de végétation augmente. Encore une fois, l'indice de végétation donne un ordre aux données.
\begin{figure}[H]
	\centering
	\caption{Probabilité de présence de Faidherbia albida suivant la dynamique de la productivité végétale entre 2000 et 2015(regroupé en 3 classes)}
	\includegraphics[height=9cm]{Slope_mean_groupe_Faidherbia_3}
\end{figure}
\begin{figure}[H]
	\centering
	\caption{Probabilité de présence de Faidherbia albida suivant la couverture ligneuse(regroupé en 3 classes)}
	\includegraphics[height=9cm]{Woody_mean_groupe_Faidherbia_3}
\end{figure}
\begin{figure}[H]
	\centering
	\caption{Probabilité de présence de Faidherbia albida suivant l’évapotranspiration réelle(regroupé en 3 classes)}
	\includegraphics[height=9cm]{AET_mean_Faidherbia_3}
\end{figure}
\begin{figure}[H]
	\centering
	\caption{Probabilité de présence de Faidherbia albida suivant le type de sol}
	\includegraphics[height=9cm]{Type_Sol_Faidherbia}
\end{figure}

%%%%%%%%%%%%%%%%%%%%%%%%%%%%%%%%%%%%%%%%% Faidherbia albida %%%%%%%%%%%%%%%%%%%
\subsection{Balanites aegyptiaca}
%%%%%%%%%%%%%%%%%%%%%%%%%%%%%%%%%%%%%%%%% Balanites aegyptiaca

\begin{figure}[H]
	\centering
	\caption{Probabilité de présence de Balanites aegyptiaca dans les différentes zones}
	\includegraphics[height=9cm]{Zone_Balanites}	
\end{figure}


\begin{figure}[H]
	\centering
	\caption{Probabilité de présence de Balanites aegyptiaca suivant l'indice de végétation(regroupé en 3 classes)}
	\includegraphics[height=9cm]{NDVI_mean_groupe_Balanites_3}
\end{figure}


\begin{figure}[H]
	\centering
	\caption{Probabilité de présence de Balanites aegyptiaca suivant le Slope(regroupé en 3 classes)}
	\includegraphics[height=9cm]{Slope_mean_groupe_Balanites_3}
\end{figure}


\begin{figure}[H]
	\centering
	\caption{Probabilité de présence de Balanites aegyptiaca suivant le Woody(regroupé en 3 classes)}
	\includegraphics[height=9cm]{Woody_mean_groupe_Balanites_3}
\end{figure}


\begin{figure}[H]
	\centering
	\caption{Probabilité de présence de Balanites aegyptiaca suivant le AET(regroupé en 3 classes)}
	\includegraphics[height=9cm]{AET_mean_groupe_Balanites_3}
\end{figure}
\begin{figure}[H]
	\centering
	\caption{Probabilité de présence de Balanites aegyptiaca suivant le type de sol}
	\includegraphics[height=9cm]{Type_Sol_Balanites}
\end{figure}
\begin{figure}[H]
	\centering
	\caption{Distribution de Balanites aegyptiaca suivant la zone 1}
	\includegraphics[height=10cm]{PA_BalanitesZ1}	
\end{figure}
\begin{figure}[H]
	\centering
	\caption{Distribution de Balanites aegyptiaca suivant la zone 2}
	\includegraphics[height=10cm]{PA_BalanitesZ2}	
\end{figure}
\begin{figure}[H]
	\centering
	\caption{Distribution de Balanites aegyptiaca suivant la zone 3}
	\includegraphics[height=10cm]{PA_BalanitesZ3}	
\end{figure}
\begin{figure}[H]
	\centering
	\caption{Distribution de Balanites aegyptiaca suivant la zone 4}
	\includegraphics[height=10cm]{PA_BalanitesZ4}	
\end{figure}
%%%%%%%%%%%%%%%%%%% 
\subsection{Anogeissus leiocarpus}
%%%%%%%%%%%%%%%%%%%%%%%%%%%%%%%%%%%%%%%%% Anogeissus leiocarpus
\begin{figure}[H]
	\centering
	\caption{Probabilité de présence de Anogeissus leiocarpus dans les différentes zones}
	\includegraphics[height=9cm]{Zone_Anogeissus}	
\end{figure}


\begin{figure}[H]
	\centering
	\caption{Probabilité de présence de Anogeissus leiocarpus suivant l'indice de végétation(regroupé en 3 classes)}
	\includegraphics[height=9cm]{NDVI_mean_groupe_Anogeissus_3}
\end{figure}


\begin{figure}[H]
	\centering
	\caption{Probabilité de présence de Anogeissus leiocarpus suivant le Slope(regroupé en 3 classes)}
	\includegraphics[height=9cm]{Slope_mean_groupe_Anogeissus_3}
\end{figure}


\begin{figure}[H]
	\centering
	\caption{Probabilité de présence de Anogeissus leiocarpus suivant le Woody(regroupé en 3 classes)}
	\includegraphics[height=9cm]{Woody_mean_groupe_Anogeissus_3}
\end{figure}


\begin{figure}[H]
	\centering
	\caption{Probabilité de présence de Anogeissus leiocarpus suivant le AET(regroupé en 3 classes)}
	\includegraphics[height=9cm]{AET_mean_groupe_Anogeissus_3}
\end{figure}
\begin{figure}[H]
	\centering
	\caption{Probabilité de présence de Anogeissus leiocarpus suivant le type de sol}
	\includegraphics[height=9cm]{Type_Sol_Anogeissus}
\end{figure}
\begin{figure}[H]
	\centering
	\caption{Distribution de Anogeissus leiocarpus suivant la zone 1}
	\includegraphics[height=10cm]{PA_AnogeissusZ1}	
\end{figure}
\begin{figure}[H]
	\centering
	\caption{Distribution de Anogeissus leiocarpus suivant la zone 2}
	\includegraphics[height=10cm]{PA_AnogeissusZ2}	
\end{figure}
\begin{figure}[H]
	\centering
	\caption{Distribution de Anogeissus leiocarpus suivant la zone 3}
	\includegraphics[height=10cm]{PA_AnogeissusZ3}	
\end{figure}
\begin{figure}[H]
	\centering
	\caption{Distribution de Anogeissus leiocarpus suivant la zone 4}
	\includegraphics[height=10cm]{PA_AnogeissusZ4}	
\end{figure}
 %%%%%%%%%%%%%%%%%%%
\subsection{Adansonia digitata}
%%%%%%%%%%%%%%%%%%%%%%%%%%%%%%%%%%%%%%%%% Adansonia digitata
\begin{figure}[H]
	\centering
	\caption{Probabilité de présence de Balanites aegyptiaca dans les différentes zones}
	\includegraphics[height=9cm]{Zone_Balanites}	
\end{figure}


\begin{figure}[H]
	\centering
	\caption{Probabilité de présence de Balanites aegyptiaca suivant l'indice de végétation(regroupé en 3 classes)}
	\includegraphics[height=9cm]{NDVI_mean_groupe_Balanites_3}
\end{figure}


\begin{figure}[H]
	\centering
	\caption{Probabilité de présence de Balanites aegyptiaca suivant le Slope(regroupé en 3 classes)}
	\includegraphics[height=9cm]{Slope_mean_groupe_Balanites_3}
\end{figure}


\begin{figure}[H]
	\centering
	\caption{Probabilité de présence de Balanites aegyptiaca suivant le Woody(regroupé en 3 classes)}
	\includegraphics[height=9cm]{Woody_mean_groupe_Balanites_3}
\end{figure}


\begin{figure}[H]
	\centering
	\caption{Probabilité de présence de Balanites aegyptiaca suivant le AET(regroupé en 3 classes)}
	\includegraphics[height=9cm]{AET_mean_groupe_Balanites_3}
\end{figure}
\begin{figure}[H]
	\centering
	\caption{Probabilité de présence de Balanites aegyptiaca suivant le type de sol}
	\includegraphics[height=9cm]{Type_Sol_Balanites}
\end{figure}
\begin{figure}[H]
	\centering
	\caption{Distribution de Adansonia digitata suivant la zone 1}
	\includegraphics[height=10cm]{PA_AdansoniAdZ1}	
\end{figure}
\begin{figure}[H]
	\centering
	\caption{Distribution de Adansonia digitata suivant la zone 2}
	\includegraphics[height=10cm]{PA_AdansoniAdZ2}	
\end{figure}
\begin{figure}[H]
	\centering
	\caption{Distribution de Adansonia digitata suivant la zone 3}
	\includegraphics[height=10cm]{PA_AdansoniAdZ3}	
\end{figure}
\begin{figure}[H]
	\centering
	\caption{Distribution de Adansonia digitata suivant la zone 4}
	\includegraphics[height=12cm]{PA_AdansoniAdZ4}	
\end{figure}
%%%%%%%%%%%%%%%%%%% 
 %%%%%%%%%%%%%%%%%%%
\subsection{Acacia nilotica}
%%%%%%%%%%%%%%%%%%%%%%%%%%%%%%%%%%%%%%%%% Acacia nilotica
\begin{figure}[H]
	\centering
	\caption{Probabilité de présence de Acacia nilotica dans les différentes zones}
	\includegraphics[height=9cm]{Zone_Balanites}	
\end{figure}


\begin{figure}[H]
	\centering
	\caption{Probabilité de présence de Acacia nilotica suivant l'indice de végétation(regroupé en 3 classes)}
	\includegraphics[height=9cm]{NDVI_mean_groupe_Balanites_3}
\end{figure}


\begin{figure}[H]
	\centering
	\caption{Probabilité de présence de Acacia nilotica suivant le Slope(regroupé en 3 classes)}
	\includegraphics[height=9cm]{Slope_mean_groupe_Balanites_3}
\end{figure}


\begin{figure}[H]
	\centering
	\caption{Probabilité de présence de Acacia nilotica suivant le Woody(regroupé en 3 classes)}
	\includegraphics[height=9cm]{Woody_mean_groupe_Balanites_3}
\end{figure}


\begin{figure}[H]
	\centering
	\caption{Probabilité de présence de Acacia nilotica suivant le AET(regroupé en 3 classes)}
	\includegraphics[height=9cm]{AET_mean_groupe_Balanites_3}
\end{figure}
\begin{figure}[H]
	\centering
	\caption{Probabilité de présence de Acacia nilotica suivant le type de sol}
	\includegraphics[height=9cm]{Type_Sol_Balanites}
\end{figure}
\begin{figure}[H]
	\centering
	\caption{Distribution de Acacia nilotica suivant la zone 1}
	\includegraphics[height=10cm]{PA_AcaciAcZ1}	
\end{figure}
\begin{figure}[H]
	\centering
	\caption{Distribution de Acacia nilotica suivant la zone 2}
	\includegraphics[height=10cm]{PA_AcaciAcZ2}	
\end{figure}
\begin{figure}[H]
	\centering
	\caption{Distribution de Acacia nilotica suivant la zone 3}
	\includegraphics[height=10cm]{PA_AcaciAcZ3}	
\end{figure}
\begin{figure}[H]
	\centering
	\caption{Distribution de Acacia nilotica suivant la zone 4}
	\includegraphics[height=10cm]{PA_AcaciAcZ4}	
\end{figure}
%%%%%%%%%%%%%%%%%%% 
 %%%%%%%%%%%%%%%%%%%
\chapter{Autocorrélation spatiale}
L’autocorrélation mesure la corrélation d’une variable avec elle-même, lorsque les observations
sont considérées avec un décalage dans le temps (autocorrélation temporelle) ou dans l’espace
(autocorrélation spatiale). On définit l’autocorrélation spatiale comme la corrélation, positive ou
négative, d’une variable avec elle-même du fait de la localisation spatiale des observations. \\
D’un point de vue statistique, de nombreuses analyses (analyse des corrélations, régressions
linéaires, etc.) reposent sur l’hypothèse d’indépendance des variables. Lorsqu’une variable est
spatialement autocorrélée, l’hypothèse d’indépendance n’est plus respectée, remettant ainsi en
cause la validité des hypothèses sur la base desquelles ces analyses sont menées. D’autre part,
l’analyse de l’autocorrélation spatiale permet une analyse quantifiée de la structure spatiale du
phénomène étudié(présence/absence d'une espèce dans notre étude).
En présence d’autocorrélation spatiale, on observe que la valeur d’une variable pour une
observation est liée aux valeurs de cette même variable pour les observations voisines.
\begin{itemize}
	\item l’autocorrélation spatiale est positive lorsque des valeurs similaires de la variable à étudier
	se regroupent géographiquement;
	\item l’autocorrélation spatiale est négative lorsque des valeurs dissemblables de la variable à
	étudier se regroupent géographiquement : des lieux proches sont plus différents que des lieux
	éloignés;
	\item en l’absence d’autocorrélation spatiale, on peut considérer que la répartition spatiale des
	observations est aléatoire.
\end{itemize}
Lorsque la variable d’intérêt n’est pas continue, mais catégorielle(Présence(P ou 1) ou Absence(A ou 0)), on mesure le degré d’association locale grâce à une analyse des statistiques des join count (ZHUKOV 2010).\\
On observe :
\begin{itemize}
	\item une autocorrélation spatiale positive si le nombre de liaisons Présence-Absence est significativement
	inférieur à ce que l’on aurait obtenu à partir d’une répartition spatiale aléatoire;
	\item une autocorrélation spatiale négative si le nombre de liaisons Présence-Absence est significativement supérieur à ce que l’on aurait obtenu à partir d’une répartition spatiale aléatoire;
	\item aucune autocorrélation spatiale si le nombre de liaisons Présence-Absence est approximativement
	identique à ce que l’on aurait obtenu à partir d’une répartition spatiale aléatoire.
\end{itemize}
S’il y a $n$ observations, $n_p$ observations de présence et $n_a = n-n_p$ observations d'absence, la probabilité de présence de l'espèce est : $P_p = \frac{n_p}{n}$
et sa probabilité d’absence
est : $P_a = 1-P_p$.\\
En l’absence d’autocorrélation spatiale, les probabilités de présence et d'absence dans deux polygones voisins sont $P_{pp}=P_pP_p=P_p^2$ et $P_{aa}=(1-P_a)(1-P_a)=(1-P_a)^2$.\\
La probabilité que l'espèce soit présente ou absente dans l'un des deux polygones est $P_{pa}=P_p(1-P_p)+P_p(1-P_p)=2P_p(1-P_p)$.\\
Sous l’hypothèse d’une répartition spatiale aléatoire des observations, $\frac{1}{2}\sum_i\sum_jw_{ij}$  mesure le nombre de liaisons existantes, on peut écrire:
\begin{eqnarray}
E[pp] = \frac{1}{2} \sum_i\sum_jw_{ij}P_p^2\\
E[aa] = \frac{1}{2} \sum_i\sum_jw_{ij}(1-P_p)^2 \\
E[pa] = \frac{1}{2} \sum_i\sum_jw_{ij}2P_p(1-P_p)
\end{eqnarray}
Si l’on désigne par $y_i = 1$ lorsque l’espèce est présente et par $y_i = 0$ dans le cas
contraire , les contre-parties empiriques (valeurs observées) de ces espérances
mathématiques peuvent s’écrire :
\begin{eqnarray}
pp = \frac{1}{2} \sum_i\sum_jw_{ij}y_iy_j\\
aa = \frac{1}{2} \sum_i\sum_jw_{ij}(1-y_i)(1-y_j) \\
pa = \frac{1}{2} \sum_i\sum_jw_{ij}(y_i-y_j)^2
\end{eqnarray}
Dans ce cas, la statistique de test permettant d’évaluer la significativité de l’autocorrélation
spatiale repose sur l’hypothèse qu’en l’absence d’autocorrélation spatiale, les statistiques de join
count ($pp$, $aa$ et $pa$) suivent une loi normale. Nous écrivons alors :
\begin{eqnarray}
\frac{pa-E[pa]}{\sqrt{var(pa)}} suit \mathit{N}(0,1)\\
\frac{pp-E[pp]}{\sqrt{var(pp)}} suit \mathit{N}(0,1)\\
\frac{aa-E[aa]}{\sqrt{var(aa)}} suit \mathit{N}(0,1)
\end{eqnarray}
\section{Autocorrélation spatiale des espèces dans la zone d'étude}
%%%%%%%%%%%%%%%%%%%%%%%%%%%%%%%%%%%%%%%%% Faidherbia albida
\subsection{Faidherbia albida}
\begin{table}[H]
	\centering
	\begin{tabular}{rrrrr}
		\hline
		& Joincount & Expected & Variance & z-value \\ 
		\hline
		0:0 & 1914.80 & 1566.58 & 79.87 & 38.96 \\ 
		1:1 & 1121.90 & 809.58 & 64.65 & 38.84 \\ 
		1:0 & 1592.30 & 2252.85 & 195.97 & -47.19 \\ 
		Jtot & 1592.30 & 2252.85 & 195.97 & -47.19 \\ 
		\hline
	\end{tabular}
\end{table}
%%%%%%%%%%%%%%%%%%%
\subsection{Balanites aegyptiaca}
%%%%%%%%%%%%%%%%%%%%%%%%%%%%%%%%%%%%%%%%% Balanites aegyptiaca
\begin{table}[H]
	\centering
	\begin{tabular}{rrrrr}
		\hline
		& Joincount & Expected & Variance & z-value \\ 
		\hline
		0:0 & 3777.00 & 3667.81 & 37.44 & 17.84 \\ 
		1:1 & 180.40 & 55.81 & 8.26 & 43.36 \\ 
		1:0 & 671.60 & 905.38 & 54.01 & -31.81 \\ 
		Jtot & 671.60 & 905.38 & 54.01 & -31.81 \\ 
		\hline
	\end{tabular}
\end{table}
 %%%%%%%%%%%%%%%%%%%
\subsection{Anogeissus leiocarpus}
%%%%%%%%%%%%%%%%%%%%%%%%%%%%%%%%%%%%%%%%% Anogeissus leiocarpus
\begin{table}[H]
	\centering
	\begin{tabular}{rrrrr}
		\hline
		& Joincount & Expected & Variance & z-value \\ 
		\hline
		0:0 & 3926.00 & 3854.39 & 30.63 & 12.94 \\ 
		1:1 & 110.00 & 35.39 & 5.44 & 32.00 \\ 
		1:0 & 593.00 & 739.21 & 41.60 & -22.67 \\ 
		Jtot & 593.00 & 739.21 & 41.60 & -22.67 \\ 
		\hline
	\end{tabular}
\end{table}
 %%%%%%%%%%%%%%%%%%%
\subsection{Adansonia digitata}
%%%%%%%%%%%%%%%%%%%%%%%%%%%%%%%%%%%%%%%%% Adansonia digitata
\begin{table}[H]
	\centering
	\begin{tabular}{rrrrr}
		\hline
		& Joincount & Expected & Variance & z-value \\ 
		\hline
		0:0 & 4146.30 & 4073.70 & 22.30 & 15.37 \\ 
		1:1 & 95.50 & 17.70 & 2.84 & 46.20 \\ 
		1:0 & 387.20 & 537.59 & 28.06 & -28.39 \\ 
		Jtot & 387.20 & 537.59 & 28.06 & -28.39 \\ 
		\hline
	\end{tabular}
\end{table}
 %%%%%%%%%%%%%%%%%%%
\subsection{Acacia nilotica}
%%%%%%%%%%%%%%%%%%%%%%%%%%%%%%%%%%%%%%%%% Acacia nilotica
\begin{table}[H]
	\centering
	\begin{tabular}{rrrrr}
		\hline
		& Joincount & Expected & Variance & z-value \\ 
		\hline
		0:0 & 4290.30 & 4278.86 & 14.23 & 3.03 \\ 
		1:1 & 29.10 & 6.86 & 1.14 & 20.81 \\ 
		1:0 & 309.60 & 343.27 & 16.56 & -8.27 \\ 
		Jtot & 309.60 & 343.27 & 16.56 & -8.27 \\ 
		\hline
	\end{tabular}
\end{table}
 %%%%%%%%%%%%%%%%%%%
\section{Autocorrélation spatiale dans chaque zone d'étude}
%%%%%%%%%%%%%%%%%%%%%%%%%%%%%%%%%%%%%%%%% Faidherbia albida
\subsection{Faidherbia albida}
%zone 1
\begin{table}[H]
	\centering
	\caption{\textit{Joincount de Faidherbia albida dans la zone 1}}
	\begin{tabular}{|r|r|r|r|r|}
		\hline
		& Joincount & Expected & Variance & z-value \\ 
		\hline
		0:0 & 61.90 & 56.66 & 1.57 & 4.18 \\ 
		\hline
		1:1 & 7.60 & 5.16 & 0.62 & 3.10 \\ 
		\hline
		1:0 & 27.00 & 34.69 & 2.63 & -4.74 \\  
		\hline
	\end{tabular}
\caption*{\textbf{Source: }Projets SERENA et LYSA(2017-2019), Calcul de l'auteur}
\end{table}
%zone 2

\begin{table}[H]
	\centering
	\caption{\textit{Joincount de Faidherbia albida dans la zone 2}}
	\begin{tabular}{|r|r|r|r|r|}
		\hline
		& Joincount & Expected & Variance & z-value \\ 
		\hline
		0:0 & 526.60 & 441.65 & 23.05 & 17.70 \\
		\hline 
		1:1 & 314.90 & 235.65 & 18.76 & 18.30 \\ 
		\hline
		1:0 & 481.50 & 645.71 & 56.11 & -21.92 \\  
		\hline
	\end{tabular}
\caption*{\textbf{Source: }Projets SERENA et LYSA(2017-2019), Calcul de l'auteur}
\end{table}
%zone 3
\begin{table}[H]
	\centering
	\caption{\textit{Joincount de Faidherbia albida dans la zone 3}}
	\begin{tabular}{|r|r|r|r|r|}
		\hline
		& Joincount & Expected & Variance & z-value \\ 
		\hline
		0:0 & 584.30 & 459.02 & 26.31 & 24.43 \\ 
		\hline
		1:1 & 422.10 & 310.02 & 23.10 & 23.32 \\ 
		\hline
		1:0 & 517.60 & 754.96 & 66.01 & -29.22 \\ 
		\hline
	\end{tabular}
\caption*{\textbf{Source: }Projets SERENA et LYSA(2017-2019), Calcul de l'auteur}
\end{table}
\begin{table}[H]
	\centering
	\caption{\textit{Joincount de Faidherbia albida dans la zone 4}}
	\begin{tabular}{|r|r|r|r|r|}
		\hline
		& Joincount & Expected & Variance & z-value \\ 
		\hline
		Absence:Absence & 739.30 & 614.73 & 29.15 & 23.07 \\ 
		\hline
		Présence:Présence & 379.70 & 264.23 & 22.21 & 24.50 \\ 
		\hline
		Présence:Absence & 566.50 & 806.55 & 69.39 & -28.82 \\  
		\hline
	\end{tabular}
\caption*{\textbf{Source: }Projets SERENA et LYSA(2017-2019), Calcul de l'auteur}
\end{table}
 %%%%%%%%%%%%%%%%%%%
\subsection{Balanites aegyptiaca}
%%%%%%%%%%%%%%%%%%%%%%%%%%%%%%%%%%%%%%%%% Balanites aegyptiaca
%zone 1
\begin{table}[ht]
	\centering
	\begin{tabular}{rrrrr}
		\hline
		& Joincount & Expected & Variance & z-value \\ 
		\hline
		0:0 & 74.40 & 73.06 & 1.06 & 1.30 \\ 
		1:1 & 3.60 & 1.56 & 0.22 & 4.30 \\ 
		1:0 & 18.50 & 21.88 & 1.45 & -2.80 \\ 
		Jtot & 18.50 & 21.88 & 1.45 & -2.80 \\ 
		\hline
	\end{tabular}
\end{table}

%zone 2
\begin{table}[ht]
	\centering
	\begin{tabular}{rrrrr}
		\hline
		& Joincount & Expected & Variance & z-value \\ 
		\hline
		0:0 & 1137.00 & 1118.54 & 8.33 & 6.40 \\ 
		1:1 & 30.80 & 8.54 & 1.33 & 19.34 \\ 
		1:0 & 155.20 & 195.93 & 10.97 & -12.30 \\ 
		Jtot & 155.20 & 195.93 & 10.97 & -12.30 \\ 
		\hline
	\end{tabular}
\end{table}

%zone 3
\begin{table}[ht]
	\centering
	\begin{tabular}{rrrrr}
		\hline
		& Joincount & Expected & Variance & z-value \\ 
		\hline
		0:0 & 1308.50 & 1284.21 & 9.93 & 7.71 \\ 
		1:1 & 35.50 & 10.21 & 1.58 & 20.11 \\ 
		1:0 & 180.00 & 229.57 & 13.04 & -13.73 \\ 
		Jtot & 180.00 & 229.57 & 13.04 & -13.73 \\ 
		\hline
	\end{tabular}
\end{table}

%zone 4
\begin{table}[H]
	\centering
	\begin{tabular}{rrrrr}
		\hline
		& Joincount & Expected & Variance & z-value \\ 
		\hline
		0:0 & 1256.30 & 1197.94 & 18.30 & 13.64 \\ 
		1:1 & 110.80 & 41.44 & 5.64 & 29.19 \\ 
		1:0 & 318.40 & 446.12 & 29.46 & -23.53 \\ 
		Jtot & 318.40 & 446.12 & 29.46 & -23.53 \\ 
		\hline
	\end{tabular}
\end{table}

 %%%%%%%%%%%%%%%%%%%
\subsection{Anogeissus leiocarpus}
%%%%%%%%%%%%%%%%%%%%%%%%%%%%%%%%%%%%%%%%% Anogeissus leiocarpus
%zone 1
\begin{table}[H]
	\centering
	\begin{tabular}{rrrrr}
		\hline
		& Joincount & Expected & Variance & z-value \\ 
		\hline
		0:0 & 64.90 & 62.16 & 1.43 & 2.29 \\ 
		1:1 & 6.60 & 3.66 & 0.47 & 4.28 \\ 
		1:0 & 25.00 & 30.68 & 2.24 & -3.80 \\ 
		Jtot & 25.00 & 30.68 & 2.24 & -3.80 \\ 
		\hline
	\end{tabular}
\end{table}
%zone 2
\begin{table}[H]
	\centering
	\begin{tabular}{rrrrr}
		\hline
		& Joincount & Expected & Variance & z-value \\ 
		\hline
		0:0 & 1094.90 & 1077.53 & 9.89 & 5.52 \\ 
		1:1 & 34.70 & 12.53 & 1.89 & 16.11 \\ 
		1:0 & 193.40 & 232.93 & 13.64 & -10.71 \\ 
		Jtot & 193.40 & 232.93 & 13.64 & -10.71 \\ 
		\hline
	\end{tabular}
\end{table}

%zone 3
\begin{table}[H]
	\centering
	\begin{tabular}{rrrrr}
		\hline
		& Joincount & Expected & Variance & z-value \\ 
		\hline
		0:0 & 1291.80 & 1260.45 & 10.84 & 9.52 \\ 
		1:1 & 43.30 & 12.45 & 1.90 & 22.37 \\ 
		1:0 & 188.90 & 251.09 & 14.57 & -16.29 \\ 
		Jtot & 188.90 & 251.09 & 14.57 & -16.29 \\ 
		\hline
	\end{tabular}
\end{table}
%zone 4
\begin{table}[H]
	\centering
	\begin{tabular}{rrrrr}
		\hline
		& Joincount & Expected & Variance & z-value \\ 
		\hline
		0:0 & 1473.10 & 1455.87 & 9.18 & 5.69 \\ 
		1:1 & 24.00 & 8.37 & 1.32 & 13.61 \\ 
		1:0 & 188.40 & 221.26 & 11.85 & -9.55 \\ 
		Jtot & 188.40 & 221.26 & 11.85 & -9.55 \\ 
		\hline
	\end{tabular}
\end{table}

 %%%%%%%%%%%%%%%%%%%
\subsection{Adansonia digitata}
%%%%%%%%%%%%%%%%%%%%%%%%%%%%%%%%%%%%%%%%% Adansonia digitata
%zone 1
\begin{table}[H]
	\centering
	\begin{tabular}{rrrrr}
		\hline
		& Joincount & Expected & Variance & z-value \\ 
		\hline
		0:0 & 81.40 & 80.21 & 0.77 & 1.36 \\ 
		1:1 & 2.90 & 0.71 & 0.11 & 6.65 \\ 
		1:0 & 12.20 & 15.58 & 0.96 & -3.45 \\ 
		Jtot & 12.20 & 15.58 & 0.96 & -3.45 \\ 
		\hline
	\end{tabular}
\end{table}
%zone 2
\begin{table}[H]
	\centering
	\begin{tabular}{rrrrr}
		\hline
		& Joincount & Expected & Variance & z-value \\ 
		\hline
		0:0 & 1198.90 & 1180.06 & 5.92 & 7.74 \\ 
		1:1 & 24.60 & 4.06 & 0.66 & 25.37 \\ 
		1:0 & 99.50 & 138.89 & 7.23 & -14.64 \\ 
		Jtot & 99.50 & 138.89 & 7.23 & -14.64 \\ 
		\hline
	\end{tabular}
\end{table}
%zone 3
\begin{table}[H]
	\centering
	\begin{tabular}{rrrrr}
		\hline
		& Joincount & Expected & Variance & z-value \\ 
		\hline
		0:0 & 1353.40 & 1327.73 & 8.21 & 8.96 \\ 
		1:1 & 36.30 & 6.73 & 1.07 & 28.62 \\ 
		1:0 & 134.30 & 189.54 & 10.32 & -17.20 \\ 
		Jtot & 134.30 & 189.54 & 10.32 & -17.20 \\ 
		\hline
	\end{tabular}
\end{table}
%zone 4
\begin{table}[H]
	\centering
	\begin{tabular}{rrrrr}
		\hline
		& Joincount & Expected & Variance & z-value \\ 
		\hline
		0:0 & 1513.30 & 1485.77 & 8.03 & 9.72 \\ 
		1:1 & 31.70 & 6.27 & 1.00 & 25.39 \\ 
		1:0 & 140.50 & 193.47 & 10.06 & -16.70 \\ 
		Jtot & 140.50 & 193.47 & 10.06 & -16.70 \\ 
		\hline
	\end{tabular}
\end{table}
 %%%%%%%%%%%%%%%%%%%
\subsection{Acacia nilotica}
%%%%%%%%%%%%%%%%%%%%%%%%%%%%%%%%%%%%%%%%% Acacia nilotica
%zone 1
\begin{table}[H]
	\centering
	\begin{tabular}{rrrrr}
		\hline
		& Joincount & Expected & Variance & z-value \\ 
		\hline
		0:0 & 89.40 & 89.61 & 0.34 & -0.36 \\ 
		1:1 & 0.70 & 0.11 & 0.02 & 4.39 \\ 
		1:0 & 6.40 & 6.78 & 0.37 & -0.62 \\ 
		Jtot & 6.40 & 6.78 & 0.37 & -0.62 \\ 
		\hline
	\end{tabular}
\end{table}

%zone 2
\begin{table}[H]
	\centering
	\begin{tabular}{rrrrr}
		\hline
		& Joincount & Expected & Variance & z-value \\ 
		\hline
		0:0 & 1225.00 & 1222.95 & 4.18 & 1.00 \\ 
		1:1 & 10.50 & 1.95 & 0.32 & 15.04 \\ 
		1:0 & 87.50 & 98.11 & 4.83 & -4.82 \\ 
		Jtot & 87.50 & 98.11 & 4.83 & -4.82 \\ 
		\hline
	\end{tabular}
\end{table}
%zone 3
\begin{table}[H]
	\centering
	\begin{tabular}{rrrrr}
		\hline
		& Joincount & Expected & Variance & z-value \\ 
		\hline
		0:0 & 1429.30 & 1425.62 & 4.20 & 1.79 \\ 
		1:1 & 8.40 & 1.62 & 0.27 & 13.00 \\ 
		1:0 & 86.30 & 96.75 & 4.74 & -4.80 \\ 
		Jtot & 86.30 & 96.75 & 4.74 & -4.80 \\ 
		\hline
	\end{tabular}
\end{table}
%zone 4
\begin{table}[H]
	\centering
	\begin{tabular}{rrrrr}
		\hline
		& Joincount & Expected & Variance & z-value \\ 
		\hline
		0:0 & 1546.10 & 1540.73 & 5.87 & 2.22 \\ 
		1:1 & 9.80 & 3.23 & 0.53 & 9.02 \\ 
		1:0 & 129.60 & 141.54 & 6.95 & -4.53 \\ 
		Jtot & 129.60 & 141.54 & 6.95 & -4.53 \\ 
		\hline
	\end{tabular}
\end{table}
 %%%%%%%%%%%%%%%%%%%


\chapter{Modélisation de la distribution des principales espèces ligneuses}
\section{Modélisation de la distribution des  espèces à partir des variables bioclimatiques}
\subsection{Les modèles}
\subsection{Réponse écologique des espèces étudiées}
%%%%%%%%%%%%%%%%%%%%%%%%%%%%%%%%%%%%%%%%% Faidherbia albida
\subsubsection{Réponse écologique de Faidherbia albida}
%%%%%%%%%%%%%%%%%%%
\subsubsection{Réponse écologique de Balanites aegyptiaca}
%%%%%%%%%%%%%%%%%%%%%%%%%%%%%%%%%%%%%%%%% Balanites aegyptiaca %%%%%%%%%%%%%%%%%%%
\subsubsection{Réponse écologique d'Anogeissus leiocarpus}
%%%%%%%%%%%%%%%%%%%%%%%%%%%%%%%%%%%%%%%%% Anogeissus leiocarpus %%%%%%%%%%%%%%%%%%%
\subsubsection{Réponse écologique d'Adansonia digitata}
%%%%%%%%%%%%%%%%%%%%%%%%%%%%%%%%%%%%%%%%% Adansonia digitata %%%%%%%%%%%%%%%%%%%
\subsubsection{Réponse écologique d'Acacia nilotica}
%%%%%%%%%%%%%%%%%%%%%%%%%%%%%%%%%%%%%%%%% Acacia nilotica %%%%%%%%%%%%%%%%%%%
\section{Modélisation de la distribution des  espèces par ajout des  variables à haute résolution}
\subsection{Les modèles}
\subsection{Réponse écologique des espèces étudiées}
%%%%%%%%%%%%%%%%%%%%%%%%%%%%%%%%%%%%%%%%% Faidherbia albida
\subsubsection{Réponse écologique de Faidherbia albida}
%%%%%%%%%%%%%%%%%%%
\subsubsection{Réponse écologique de Balanites aegyptiaca}
%%%%%%%%%%%%%%%%%%%%%%%%%%%%%%%%%%%%%%%%% Balanites aegyptiaca %%%%%%%%%%%%%%%%%%%
\subsubsection{Réponse écologique d'Anogeissus leiocarpus}
%%%%%%%%%%%%%%%%%%%%%%%%%%%%%%%%%%%%%%%%% Anogeissus leiocarpus %%%%%%%%%%%%%%%%%%%
\subsubsection{Réponse écologique d'Adansonia digitata}
%%%%%%%%%%%%%%%%%%%%%%%%%%%%%%%%%%%%%%%%% Adansonia digitata %%%%%%%%%%%%%%%%%%%
\subsubsection{Réponse écologique d'Acacia nilotica}
%%%%%%%%%%%%%%%%%%%%%%%%%%%%%%%%%%%%%%%%% Acacia nilotica %%%%%%%%%%%%%%%%%%%
\subsection{Distributions futures des espèces}
%%%%%%%%%%%%%%%%%%%%%%%%%%%%%%%%%%%%%%%%% Faidherbia albida
\subsubsection{Distribution de Faidherbia albida}
%%%%%%%%%%%%%%%%%%%
\subsubsection{Distribution de Balanites aegyptiaca}
%%%%%%%%%%%%%%%%%%%%%%%%%%%%%%%%%%%%%%%%% Balanites aegyptiaca %%%%%%%%%%%%%%%%%%%
\subsubsection{Distribution d'Anogeissus leiocarpus}
%%%%%%%%%%%%%%%%%%%%%%%%%%%%%%%%%%%%%%%%% Anogeissus leiocarpus %%%%%%%%%%%%%%%%%%%
\subsubsection{Distribution d'Adansonia digitata}
%%%%%%%%%%%%%%%%%%%%%%%%%%%%%%%%%%%%%%%%% Adansonia digitata %%%%%%%%%%%%%%%%%%%
\subsubsection{Distribution d'Acacia nilotica}
%%%%%%%%%%%%%%%%%%%%%%%%%%%%%%%%%%%%%%%%% Acacia nilotica %%%%%%%%%%%%%%%%%%%
Le graphique donne une vue d'ensemble des variables environnementales c'est-à-dire la quantité de zéros, de valeurs manquantes(NA), de valeurs uniques, ainsi que le type de données. Ces informations sur les données peuvent conduire à un bon ou mauvais modèle.
\begin{itemize}
	\item \textbf{q\_zeros}: le nombre de zéros ( \textbf{p\_zeros}: en pourcentage);
	\item \textbf{q\_na}: le nombre de valeurs manquantes ( \textbf{p\_na}: en pourcentage);
	\item \textbf{type}: facteur ou numérique;
	\item \textbf{unique}: quantité de valeurs uniques.
\end{itemize}
Ces mesures sont importantes pour plusieurs raisons:
\begin{itemize}
	\item \textbf{Zéros} : les variables avec beaucoup de zéros peuvent ne pas être utiles pour la modélisation, et dans certains cas, elles peuvent biaiser considérablement le modèle;
	\item \textbf{NA} : plusieurs modèles excluent automatiquement les lignes avec NA ( forêt aléatoire , par exemple). Par conséquent, le modèle final peut être biaisé en raison de plusieurs lignes manquantes en raison d'une seule variable. Par exemple, si les données ne contiennent qu'une variable sur 100 avec 90\% des NA, le modèle s'entraînera avec seulement 10\% des lignes originales;
	\item \textbf{Type} : Certaines variables sont codées sous forme de nombres, mais ce sont des codes ou des catégories, et les modèles ne les gèrent pas de la même manière;
	\item \textbf{Unique} : les variables factorielles / catégorielles avec un nombre élevé de valeurs différentes (~ 30) ont tendance à sur-ajuster si les catégories sont peu représentatives ( arbre de décision , par exemple).
\end{itemize}
% Table generated by Excel2LaTeX from sheet 'Feuil1'
\begin{table}[htbp]
	\centering
	\caption{Add caption}
	\begin{tabular}{|r|r|r|}
		\toprule
		\textbf{Code} & \multicolumn{1}{c}{\textbf{Variables }} & Unité \\
		\midrule
		BIO1  & température moyenne annuelle & $^{\circ}C$ \\
		BIO2  & plage diurne moyenne (moyenne mensuelle (temp max - temp min)) & $^{\circ}C$ \\
		BIO3  &  isotherme (BIO2 / BIO7) (* 100) &  \\
		BIO4  & saisonnalité de la température (écart-type * 100) & $^{\circ}C$ \\
		BIO5  & température max du plus chaud Mois & $^{\circ}C$ \\
		BIO6  & Température minimale du mois le plus froid & $^{\circ}C$ \\
		BIO7  & Plage annuelle de température (BIO5-BIO6) & $^{\circ}C$ \\
		BIO8  & Température moyenne du trimestre le plus humide & $^{\circ}C$ \\
		BIO9  & Température moyenne du trimestre le plus sec & $^{\circ}C$ \\
		BIO10 & Température moyenne du trimestre le plus chaud & $^{\circ}C$ \\
		BIO11 & Température moyenne du trimestre le plus froid & $^{\circ}C$ \\
		BIO12 & Précipitations annuelles & mm \\
		BIO13 & Précipitation du mois le plus humide  & mm \\
		BIO14 & Précipitation du mois le plus sec & mm \\
		BIO15 & Saisonnalité des précipitations (coefficient de variation) &  \\
		BIO16 & Précipitation du trimestre le plus humide & mm \\
		BIO17 & Précipitation du trimestre le plus sec & mm \\
		BIO18 & Précipitation du trimestre le plus chaud & mm \\
		BIO19 & Précipitation du trimestre le plus froid & mm \\
		\bottomrule
	\end{tabular}%
	\label{tab:addlabel}%
\end{table}%
\part{Discussion}








\newpage
\begin{thebibliography}{99}



\end{thebibliography}



\begin{appendix}
\setcounter{page}{1}
\pagenumbering{roman}

\chapter{Lois de probabilité usuelle}
\label{Lois de probabilité}
Cette annexe donne la fonction de densité, l'espérance mathématique, la variance, le coefficient d’asymétrie, le coefficient d’aplatissement normalisé et l'utilisation la plus répandue des distributions utilisées dans notre travail.\\
Plusieurs caractéristiques de la loi de distribution sont d’intérêt, nous nous attarderons sur quatre d’entre eux:
\begin{itemize}
	\item l’espérance mathématiques $E[X]$, qui représente la moyenne pondérée des valeurs de $X$;
	\item la variance $\sigma^2$, qui symbolise la mesure de dispersion de $X$;
	\item le coefficient d’asymétrie (ou skewness ou $\beta_1$ de Pearson) de formule $\gamma_1 = \frac{E[(X - E[X]) ^3]}{\sigma ^3}$ qui correspond à la qualité de symétrie de la distribution de $X$;
	\item le coefficient d’aplatissement normalisé (ou kurtosis ou $\beta_2$ de Pearson) de formule $\gamma_2 = \gamma_2 ' - 3 = \frac{E[(X - E[X]) ^4]}{\sigma ^4} - 3$ qui mesure le comportement des « queues » de la distribution de X.	
\end{itemize}
Dans le cas continue, en définissant $f$ la densité de X,  nous avons les formules générales suivantes: 
$$
P_X (I) = \int_I f(x) dx
$$
$$
E[X] = \int_I x \cdot f(x) dx
$$
$$
var (X) = E[X ^2] - (E[X])^2 = \int_I (x - E[X])^2 f(x) dx
$$

\chapter{Définitions}
\label{Définitions}
\chapter{Notations}
\label{Notations}
\chapter{Tests statistiques}
\label{Tests statistiques}
\section*{ok}
\section*{ok}
\chapter{Analyse spatiale descriptive}
\label{Analyse spatiale descriptive}




\chapter{Quelques démonstrations}
\label{whatever}
\begin{figure}[H]
	\centering
	\caption{ok}
	\includegraphics[height=15cm]{etat}
\end{figure}
\begin{figure}[H]
	\centering
	\caption{ok}
	\includegraphics[height=24cm,width=9cm]{tab_espece}
\end{figure}








\end{appendix}


\tableofcontents
\end{document}


\begin{MyBlock}{Résumé}
\end{MyBlock}

Une fois qu'un modèle de régression logistique a été ajusté à un ensemble de données donné, son adéquation est examinée au moyen de tests de validité globale, d'une zone sous la courbe caractéristique de fonctionnement du récepteur et d'un examen d'observations déterminantes. Le but de tout test global de qualité de l'ajustement est de déterminer si le modèle ajusté décrit de manière adéquate l'expérience de résultat observé dans les données (Hosmer et Lemeshow, 2000). On en conclut qu'un modèle convient si les différences entre les valeurs observées et ajustées sont faibles et s'il n'y a pas de contribution systématique des différences à la structure d'erreur du modèle.\\

Fondée sur l'enquête démographique et de
santé du Niger de 2012, l’analyse des tendances de la mortalité des enfants de moins de cinq ans a permis de formuler trois conclusions principales.\\
Premièrement,


\begin{comment}







\end{comment}




\begin{minipage}{.46\textwidth}
\begin{flushleft}


\end{flushleft}
\end{minipage}
\hfill
\begin{minipage}{.46\textwidth}
\begin{flushright}


\end{flushright}
\end{minipage}